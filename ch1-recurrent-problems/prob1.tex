% \begin{tikzpicture}[
% roundnode/.style={circle, draw=green!60, fill=green!5, very thick, minimum size=7mm},
% squarednode/.style={rectangle, draw=red!60, fill=red!5, very thick, minimum size=5mm},
% ]
% %Nodes
% \node[squarednode]      (maintopic)                              {2};
% \node[roundnode]        (uppercircle)       [above=of maintopic] {1};
% \node[squarednode]      (rightsquare)       [right=of maintopic] {3};
% \node[roundnode]        (lowercircle)       [below=of maintopic] {4};

% %Lines
% \draw[->] (uppercircle.south) -- (maintopic.north);
% \draw[->] (maintopic.east) -- (rightsquare.west);
% \draw[->] (rightsquare.south) .. controls +(down:7mm) and +(right:7mm) .. (lowercircle.east);
% \end{tikzpicture}

\subsection{Problem 1(1.5)} 
\textbf{A "Venn diagram" with three overlapping circles is often used to illustrate
the eight possible subsets associated with three given sets.Can the sixteen possibilities that arise with four given sets be illustrated by four overlapping circles?}

\par

\begin{flushleft}
\textbf{Solution: } \par
Let $T_n$ be the number of subsets illustrated by $n$ overlapping circles.\\
Now,
$$
\begin{array}{l}
T_1=2 \\
T_2=4=2+1 \times 2=T_1+1 \times 2 \\
T_3=8=4+2 \times 2=T_2+2 \times 2 \\
\vdots \\
T_n=T_{n-1}+(n-1) \times 2
\end{array}
$$
So for 4 overlapping circles,
$$
T_4=T_3+(4-1) \times 2=8+6=14
$$
We can see 4 overlapping circles at max can illustrate 14 possibilities (subsets).
Therefore, sixteen possibilities that arise with four given sets can't be illustrated by four overlapping
circles.
\end{flushleft}
