\subsection{Problem 2(2.12)}
\textbf{Show that the function $p(k)=k+(-1)^k c$ is a permutation of the set of all integers; whenever $c$ is an integer.}
\par

\begin{flushleft}
\textbf{Solution: }
\par
Let,
$$
\begin{aligned}
\text { Let, } & p(k)=n \in \mathbb{N} \\
\Rightarrow & k+(-1)^k c=n \quad . . .(i) \\
\Rightarrow & k+(-1)^k c+c=n+c \\
\Rightarrow & k+\left\{(-1)^k+1\right\} c=n+c
\end{aligned}
$$
Now,
$$
\begin{aligned}
& (-1)^{k+\left\{(-1)^k+1\right\} c}=(-1)^{n+c} \\
\Rightarrow & (-1)^k \cdot (-1)^{\left\{(-1)^k+1\right\} c}=(-1)^{n+c} \\
\Rightarrow & (-1)^k \cdot (-1)^{\text {even }}=(-1)^{n+c} \\
\Rightarrow & (-1)^k=(-1)^{n+c}
\end{aligned}
$$
From (i),
$$
\begin{aligned}
& n=k+(-1)^k c \\
\Rightarrow & n=k+(-1)^{n+c} c \\
\Rightarrow & k=n-(-1)^{n+c} c
\end{aligned}
$$
So, whenever $c$ is an integer, the value of $k$ above will give $p(k)=n \in \mathbb{N}$.
Therefore, the function $p(k)=k+(-1)^k c$ is a permutation of the set of all integers; whenever $c$ is an integer.
\end{flushleft}


